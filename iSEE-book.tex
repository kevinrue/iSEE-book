\PassOptionsToPackage{unicode=true}{hyperref} % options for packages loaded elsewhere
\PassOptionsToPackage{hyphens}{url}
%
\documentclass[]{book}
\usepackage{lmodern}
\usepackage{amssymb,amsmath}
\usepackage{ifxetex,ifluatex}
\usepackage{fixltx2e} % provides \textsubscript
\ifnum 0\ifxetex 1\fi\ifluatex 1\fi=0 % if pdftex
  \usepackage[T1]{fontenc}
  \usepackage[utf8]{inputenc}
  \usepackage{textcomp} % provides euro and other symbols
\else % if luatex or xelatex
  \usepackage{unicode-math}
  \defaultfontfeatures{Ligatures=TeX,Scale=MatchLowercase}
\fi
% use upquote if available, for straight quotes in verbatim environments
\IfFileExists{upquote.sty}{\usepackage{upquote}}{}
% use microtype if available
\IfFileExists{microtype.sty}{%
\usepackage[]{microtype}
\UseMicrotypeSet[protrusion]{basicmath} % disable protrusion for tt fonts
}{}
\IfFileExists{parskip.sty}{%
\usepackage{parskip}
}{% else
\setlength{\parindent}{0pt}
\setlength{\parskip}{6pt plus 2pt minus 1pt}
}
\usepackage{hyperref}
\hypersetup{
            pdftitle={Extending iSEE},
            pdfauthor={Kevin Rue-Albrecht, Federico Marini, Charlotte Soneson, and Aaron Lun},
            pdfborder={0 0 0},
            breaklinks=true}
\urlstyle{same}  % don't use monospace font for urls
\usepackage{longtable,booktabs}
% Fix footnotes in tables (requires footnote package)
\IfFileExists{footnote.sty}{\usepackage{footnote}\makesavenoteenv{longtable}}{}
\usepackage{graphicx,grffile}
\makeatletter
\def\maxwidth{\ifdim\Gin@nat@width>\linewidth\linewidth\else\Gin@nat@width\fi}
\def\maxheight{\ifdim\Gin@nat@height>\textheight\textheight\else\Gin@nat@height\fi}
\makeatother
% Scale images if necessary, so that they will not overflow the page
% margins by default, and it is still possible to overwrite the defaults
% using explicit options in \includegraphics[width, height, ...]{}
\setkeys{Gin}{width=\maxwidth,height=\maxheight,keepaspectratio}
\setlength{\emergencystretch}{3em}  % prevent overfull lines
\providecommand{\tightlist}{%
  \setlength{\itemsep}{0pt}\setlength{\parskip}{0pt}}
\setcounter{secnumdepth}{5}
% Redefines (sub)paragraphs to behave more like sections
\ifx\paragraph\undefined\else
\let\oldparagraph\paragraph
\renewcommand{\paragraph}[1]{\oldparagraph{#1}\mbox{}}
\fi
\ifx\subparagraph\undefined\else
\let\oldsubparagraph\subparagraph
\renewcommand{\subparagraph}[1]{\oldsubparagraph{#1}\mbox{}}
\fi

% set default figure placement to htbp
\makeatletter
\def\fps@figure{htbp}
\makeatother

\usepackage{etoolbox}
\makeatletter
\providecommand{\subtitle}[1]{% add subtitle to \maketitle
  \apptocmd{\@title}{\par {\large #1 \par}}{}{}
}
\makeatother
\usepackage{booktabs}
% https://github.com/rstudio/rmarkdown/issues/337
\let\rmarkdownfootnote\footnote%
\def\footnote{\protect\rmarkdownfootnote}

% https://github.com/rstudio/rmarkdown/pull/252
\usepackage{titling}
\setlength{\droptitle}{-2em}

\pretitle{\vspace{\droptitle}\centering\huge}
\posttitle{\par}

\preauthor{\centering\large\emph}
\postauthor{\par}

\predate{\centering\large\emph}
\postdate{\par}
\usepackage[]{natbib}
\bibliographystyle{apalike}

\title{Extending \emph{iSEE}}
\author{Kevin Rue-Albrecht, Federico Marini, Charlotte Soneson, and Aaron Lun}
\date{2019-12-09}

\begin{document}
\maketitle

{
\setcounter{tocdepth}{1}
\tableofcontents
}
\hypertarget{preface}{%
\chapter*{Preface}\label{preface}}
\addcontentsline{toc}{chapter}{Preface}

The \href{https://bioconductor.org/}{Bioconductor} \emph{\href{https://bioconductor.org/packages/3.11/iSEE}{iSEE}} package provides functions for creating an interactive graphical user interface (GUI) using the \href{https://rstudio.com/}{RStudio} \emph{\href{https://CRAN.R-project.org/package=Shiny}{Shiny}} package for exploring data stored in \emph{\href{https://bioconductor.org/packages/3.11/SummarizedExperiment}{SummarizedExperiment}} objects, including row- and column-level metadata \citep{rue2018isee}.
In this book we describe how to create web-applications that leverage built-in panels and develop new ones.

\hypertarget{panel-classes}{%
\chapter{Panel classes}\label{panel-classes}}

\hypertarget{overview}{%
\section{Overview}\label{overview}}

The types of panels available to compose an \emph{\href{https://bioconductor.org/packages/3.11/iSEE}{iSEE}} app are defined as a hierarchy of S4 classes.

\begin{itemize}
\tightlist
\item
  \texttt{Panel}

  \begin{itemize}
  \tightlist
  \item
    \texttt{DotPlot}

    \begin{itemize}
    \tightlist
    \item
      \texttt{ColumnDotPlot}

      \begin{itemize}
      \tightlist
      \item
        \texttt{RedDimPlot}
      \item
        \texttt{ColDataPlot}
      \item
        \texttt{FeatAssayPlot}
      \end{itemize}
    \item
      \texttt{RowDotPlot}

      \begin{itemize}
      \tightlist
      \item
        \texttt{RowDataPlot}
      \item
        \texttt{SampAssayPlot}
      \end{itemize}
    \end{itemize}
  \item
    \texttt{Table}

    \begin{itemize}
    \tightlist
    \item
      \texttt{RowTable}

      \begin{itemize}
      \tightlist
      \item
        \texttt{RowStatTable}
      \end{itemize}
    \item
      \texttt{ColumnTable}

      \begin{itemize}
      \tightlist
      \item
        \texttt{ColStatTable}
      \end{itemize}
    \end{itemize}
  \item
    \texttt{HeatMapPlot}
  \end{itemize}
\end{itemize}

\hypertarget{the-panel-class}{%
\section{The Panel class}\label{the-panel-class}}

The top-most class is called \texttt{Panel}.
It is a virtual class that defines the core properties common to any panel - existing or future - that may be displayed in the interface.

\begin{longtable}[]{@{}ll@{}}
\toprule
\endhead
\begin{minipage}[t]{0.47\columnwidth}\raggedright
\texttt{PanelId}\strut
\end{minipage} & \begin{minipage}[t]{0.47\columnwidth}\raggedright
Integer index indicating the i\textsuperscript{th} panel of a given type.\strut
\end{minipage}\tabularnewline
\begin{minipage}[t]{0.47\columnwidth}\raggedright
\texttt{PanelHeight}\strut
\end{minipage} & \begin{minipage}[t]{0.47\columnwidth}\raggedright
Height of the panel, in pixels.\strut
\end{minipage}\tabularnewline
\begin{minipage}[t]{0.47\columnwidth}\raggedright
\texttt{PanelWidth}\strut
\end{minipage} & \begin{minipage}[t]{0.47\columnwidth}\raggedright
Width of the panel, an integer value indicating the number of columns to use, from 1 to 12.\strut
\end{minipage}\tabularnewline
\begin{minipage}[t]{0.47\columnwidth}\raggedright
\texttt{SelectBoxOpen}\strut
\end{minipage} & \begin{minipage}[t]{0.47\columnwidth}\raggedright
Logical value indicating if the \emph{Selection parameters} box of the panel is open when the app starts.\strut
\end{minipage}\tabularnewline
\begin{minipage}[t]{0.47\columnwidth}\raggedright
\texttt{SelectByPlot}\strut
\end{minipage} & \begin{minipage}[t]{0.47\columnwidth}\raggedright
Encoded name of the panel from which to receive a selection of data points.\strut
\end{minipage}\tabularnewline
\begin{minipage}[t]{0.47\columnwidth}\raggedright
\texttt{SelectMultiType}\strut
\end{minipage} & \begin{minipage}[t]{0.47\columnwidth}\raggedright
Keyword indicating the method to deal with multiple incoming selections of data points.\strut
\end{minipage}\tabularnewline
\begin{minipage}[t]{0.47\columnwidth}\raggedright
\texttt{SelectMultiSaved}\strut
\end{minipage} & \begin{minipage}[t]{0.47\columnwidth}\raggedright
Integer index indicating a single data point selection to use, among multiple incoming selections.\strut
\end{minipage}\tabularnewline
\bottomrule
\end{longtable}

\hypertarget{the-dotplot-and-table-panel-families}{%
\section{The DotPlot and Table panel families}\label{the-dotplot-and-table-panel-families}}

The \texttt{Panel} virtual class is directly derived into two major virtual sub-classes:

\begin{itemize}
\tightlist
\item
  DotPlot
\item
  Table
\end{itemize}

Those classes introduce properties that are specific to distinct subsets of panel types.

The \texttt{DotPlot} class introduce parameters specific to panels where the output is a \texttt{ggplot} object and each row in the data-frame is represented as a point in a plot.

The \texttt{Table} class introduce parameters specific to panels where the main output is a data-frame directly displayed as a table in the GUI.

In addition, the \texttt{HeatMapPlot} class defines a special panel class that directly extends the \texttt{Panel} class, as it introduces a set of parameters distinct from both the \texttt{DotPlot} and \texttt{Table} panel families.
This panel type is described in further details in a separate section \protect\hyperlink{heatmapplot-class}{below}.

\hypertarget{the-columndotplot-and-rowdotplot-panel-families}{%
\section{The ColumnDotPlot and RowDotPlot panel families}\label{the-columndotplot-and-rowdotplot-panel-families}}

\hypertarget{built-in-columndotplot-panel-classes}{%
\section{Built-in ColumnDotPlot panel classes}\label{built-in-columndotplot-panel-classes}}

\hypertarget{built-in-rowdotplot-panel-classes}{%
\section{Built-in RowDotPlot panel classes}\label{built-in-rowdotplot-panel-classes}}

\hypertarget{the-columntable-and-rowtable-panel-families}{%
\section{The ColumnTable and RowTable panel families}\label{the-columntable-and-rowtable-panel-families}}

\hypertarget{built-in-columntable-panel-classes}{%
\section{Built-in ColumnTable panel classes}\label{built-in-columntable-panel-classes}}

\hypertarget{built-in-rowtable-panel-classes}{%
\section{Built-in RowTable panel classes}\label{built-in-rowtable-panel-classes}}

\hypertarget{heatmapplot-class}{%
\section{The HeatMapPlot panel class}\label{heatmapplot-class}}

This type of panel introduces parameters specific to panels where the output is a heat map, with each row representing a feature and each column representing a sample in the \texttt{se} object.

\hypertarget{server}{%
\chapter{The iSEE server}\label{server}}

\hypertarget{robjects}{%
\section{Reactive objects}\label{robjects}}

\hypertarget{persistent-non-reactive-objects}{%
\section{Persistent (non-reactive) objects}\label{persistent-non-reactive-objects}}

\hypertarget{memory}{%
\section{The app memory}\label{memory}}

The app \texttt{memory} is a list of instances created from available panel classes, which defines the order in which individual panels are displayed in the GUI.

\hypertarget{panel-api}{%
\section{The panel API}\label{panel-api}}

\hypertarget{cachecommoninfo}{%
\subsection{.cacheCommonInfo}\label{cachecommoninfo}}

Each individual panel (e.g., \emph{RedDimPlot}) and family of panels (e.g., \emph{ColDotPlot}) defines a \texttt{.cacheCommonInfo} function.

This function is called for each panel instance in memory when the app is initialized.
It allows the app to efficienly compute a single time common information that only depends on the input \texttt{se} object, and may be frequently reused during the runtime of the app.

Following the hierarchy of panel types, each call to the signature takes a panel instance \texttt{x} and the \texttt{se} object, and caches the computed information in the \texttt{se} object itself, before calling \texttt{callNextMethod()} to invoke the next parent signature.

The top-most signature - for the \texttt{Panel} class - returns the \texttt{se} object that contains all the cached information.

\hypertarget{refineparameters}{%
\subsection{.refineParameters}\label{refineparameters}}

Each individual panel (e.g., \emph{RedDimPlot}) and family of panels (e.g., \emph{ColDotPlot}) defines a \texttt{.refineParameters} function.

This function is called for each panel instance in memory when the app is initialized, and also when a new panel is added to the GUI.
It inspects the parameters of a given panel instance, and replaces invalid parameters with sensible values for a given \texttt{se} object.

Following the hierarchy of panel types, each call to the signature takes an instance \texttt{x} and the \texttt{se} object, and first calls \texttt{callNextMethod()} to invoke the next parent signature, to refine generic parameters before processing specific ones.

The called signature ultimately returns the updated instance panel \texttt{x}, or \texttt{NULL} if the panel instance is not available for this app.

\hypertarget{initialization-of-the-app-server}{%
\section{Initialization of the app server}\label{initialization-of-the-app-server}}

The app server is initialized as soon as a valid \texttt{se} object is provided.
This can be either in the call to \texttt{iSEE(se)} or using the Shiny file upload button in apps that were launched without providing the \texttt{se} arguments, e.g., \texttt{iSEE()}.

The \texttt{initialize\_server} function takes the \texttt{se} object and the list holding reactive values used to trigger re-rendering of the GUI, as described \protect\hyperlink{robjects}{above}.

The very first step invokes the function \texttt{.sanitize\_SE\_input} on the \texttt{se} object.
This function coerces the \texttt{se} to \texttt{SingleCellExperiment}, flatten nested DataFrames, add row and column names, and remove other non-atomic fields.
In addition, it also sanitizes the \texttt{SingleCellExperiment} object by moving internal fields into the column- or row-level metadata, making them visible in the \emph{Column statistics table} and \emph{Row statistics table} panels, respectively. The function returns both the sanitized \texttt{se} object that will be used by the app, and the list of R commands that will be displayed in the code tracker for users.

Next, the server invokes the \texttt{checkColormapCompatibility} function.
This function takes the \texttt{se} object and the optioanl \texttt{colormap} provided to \texttt{iSEE()}, and carries out a number of compatibility checks between the two objects.
The function collects a character vector of issue messages that are displayed - if any - as warning messages in GUI during initialization.

Next, the \texttt{.cacheCommonInfo} and \texttt{.refineParameters} are successively invoked on each panel instance initialized in the app memory.
As described in a separate section \protect\hyperlink{panel-api}{above}, the first function precomputes and caches information specific to the \texttt{se} object and frequently used throughout the runtime of the app.
The second function ensures that each panel instance is initialized with valid parameters; it replaces any invalid parameters with sensible values for a given \texttt{se} object.

Next, persistent (non-reactive) objects are initialized:

\begin{itemize}
\tightlist
\item
  the app \texttt{memory} (see this \protect\hyperlink{memory}{section})
\item
  the count of panels of each type, used to assign increasing ID to new panel instances
\item
  the list of commands to display in the code tracker for each panel instance
\item
  the list of data point coordinates selectable in each panel instance\footnote{Data points downsampled for rendering speed performance remain selectable, even though they are not visible in the plot.}
\item
  a list of miscellaneous cached information\footnote{The plot that contain the legend keys of \emph{Heatmap} panels is currently cached as miscellaneous information retrieved separately when rendering the GUI.}
\end{itemize}

\hypertarget{the-plotting-api}{%
\chapter{The plotting API}\label{the-plotting-api}}

\hypertarget{getplottingfunction}{%
\section{.getPlottingFunction}\label{getplottingfunction}}

Each panel type available for use in the GUI defines a \texttt{.getPlottingFunction}.

This function is called within \texttt{.createRenderedOutput}, which is triggered by observers when the value of the panel input widgets are changed by users, or when a new panel is added to the GUI.

The \texttt{.getPlottingFunction} function inspects the parameters for a given panel instance, and uses the app \texttt{memory} of all active panels and parameters, the coordinates of data points in each plot panel, the \texttt{se} object, and the \texttt{colormap} to generate all the information necessary to render the outputs of this panel and those that depend on it.

For \texttt{DotPlot} panels, the output is a list that includes:

\begin{itemize}
\tightlist
\item
  the list of commands to display in the code tracker
\item
  the coordinates of data points in the plot
\item
  the \texttt{ggplot} object
\end{itemize}

For \texttt{Table} panels, the output is a \texttt{datatable}.

For the \texttt{HeatMap} panel, the function does not return any value.
Instead it sets relevant elements in the \texttt{output} object of the Shiny session.

\bibliography{book.bib,packages.bib}

\end{document}
